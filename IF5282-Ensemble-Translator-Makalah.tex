\documentclass[conference]{IEEEtran}
\IEEEoverridecommandlockouts
% The preceding line is only needed to identify funding in the first footnote. If that is unneeded, please comment it out.
\usepackage[bahasa]{babel}
\usepackage{cite}
\usepackage{amsmath,amssymb,amsfonts}
\usepackage{algorithmic}
\usepackage{graphicx}
\usepackage{textcomp}
\def\BibTeX{{\rm B\kern-.05em{\sc i\kern-.025em b}\kern-.08em
    T\kern-.1667em\lower.7ex\hbox{E}\kern-.125emX}}
\begin{document}

\title{Penerjemah Ensembel Bahasa Indonesia ke Bahasa Inggris\\
}

\author{\IEEEauthorblockN{Stefanus Agus Haryono}
\IEEEauthorblockA{\textit{Sekolah Teknik Elektro dan Informatika} \\
\textit{Institut Teknologi Bandung}\\
Bandung, Indonesia \\
agus@email}
\and
\IEEEauthorblockN{Ratnadira Widyasari}
\IEEEauthorblockA{\textit{Sekolah Teknik Elektro dan Informatika} \\
\textit{Institut Teknologi Bandung}\\
Bandung, Indonesia \\
dira@email}
\and
\IEEEauthorblockN{Muhammad Nizami}
\IEEEauthorblockA{\textit{Sekolah Teknik Elektro dan Informatika} \\
\textit{Institut Teknologi Bandung}\\
Bandung, Indonesia \\
23517044@std.stei.itb.ac.id
}
}

\maketitle

\begin{abstract}
Ringkasan di sini...
\end{abstract}

\begin{IEEEkeywords}
SMT, NMT, ensembel, penilai tata pahasa
\end{IEEEkeywords}

\section{Pendahuluan}
Latar belakang masalah

\section{Penerjemah Mesin berbasis Statistik dan Jaringan Syaraf Tiruan}

\section{Penerjemah Mesin Ensembel}

\section{Penilai Tata Bahasa}

\section{Eksperimen}

\subsection{Eksperimen Penilai Tata Bahasa}

\subsection{Eksperimen Penerjemah}

\section{Kesimpulan dan Saran}

\section*{Ucapan Terima Kasih}

\begin{thebibliography}{00}
\bibitem{b11} jangan lupa ini diganti
\bibitem{b1} G. Eason, B. Noble, and I. N. Sneddon, ``On certain integrals of Lipschitz-Hankel type involving products of Bessel functions,'' Phil. Trans. Roy. Soc. London, vol. A247, pp. 529--551, April 1955.
\bibitem{b2} J. Clerk Maxwell, A Treatise on Electricity and Magnetism, 3rd ed., vol. 2. Oxford: Clarendon, 1892, pp.68--73.
\bibitem{b3} I. S. Jacobs and C. P. Bean, ``Fine particles, thin films and exchange anisotropy,'' in Magnetism, vol. III, G. T. Rado and H. Suhl, Eds. New York: Academic, 1963, pp. 271--350.
\bibitem{b4} K. Elissa, ``Title of paper if known,'' unpublished.
\bibitem{b5} R. Nicole, ``Title of paper with only first word capitalized,'' J. Name Stand. Abbrev., in press.
\bibitem{b6} Y. Yorozu, M. Hirano, K. Oka, and Y. Tagawa, ``Electron spectroscopy studies on magneto-optical media and plastic substrate interface,'' IEEE Transl. J. Magn. Japan, vol. 2, pp. 740--741, August 1987 [Digests 9th Annual Conf. Magnetics Japan, p. 301, 1982].
\bibitem{b7} M. Young, The Technical Writer's Handbook. Mill Valley, CA: University Science, 1989.
\end{thebibliography}

\end{document}
